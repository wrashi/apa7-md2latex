\hypertarget{markdown}{%
\section{Markdown}\label{markdown}}

\hypertarget{simple-structure}{%
\subsection{Simple Structure}\label{simple-structure}}

Text in Markdown uses \texttt{\#} at the beginning of a line to indicate
headings. \texttt{\#} = heading. \texttt{\#\#} = sub-heading.
\texttt{\#\#\#} = sub-sub-heading, and so on.

\hypertarget{on-sentences}{%
\subsection{On Sentences}\label{on-sentences}}

Text does not have to be formated in traditional paragraph form. LaTeX
will properly format blocks of text that are separated by more than 2
lines.

Having each sentence start at the beginning of a line, is a useful in
many ways. It shows paragraph structure more clearly. It shows how long
your sentences are. It makes it easy to find the right sentence when
editing, It is easy to shift a sentence into a new position in the
paragraph.

Putting different elements of a sentence on separate lines is also
possible. This can highlight sentence structure, build confidence that
the commas are in the right place, and still compile into a normal
looking sentence.

\hypertarget{using}{%
\section{\texorpdfstring{Using \LaTeX}{Using }}\label{using}}

\textcite{lamport1994latex} is one of the first, and still most widely
cited books on LaTeX. With some LaTeX knowledge, sprinkle it throughout
the document provides great effect. Math like \(e=mc^2\), for example,
looks terrific when formatted in LaTeX.
